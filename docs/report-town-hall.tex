\section{La mairie d'Arcachon}\label{sec:la-mairie-d'arcachon}

Arcachon est une commune du Sud-Ouest de la France, sous-préfecture du département de la Gironde, en région Nouvelle-Aquitaine.
Elle a été créée en 1857 par détachement d'une partie de la commune de La Teste-de-Buch.

La mairie contient environ 80 agents, répartis entre plusieurs services, ayant à répondre à un grand nombre de problématique (organisation des retraites, état civil, inscriptions scolaires, gestion informatique, etc).
La majorité de ces services ont en commun de récolter des demandes de la part des habitants, le plus souvent sous la forme de formulaires papier.
Par exemple, on retrouve les demandes de permis de construire, des suggestions pour la vie politique de la ville, l'inscription à la cantine de la garderie, etc.

Dans ce document, nous allons nous concentrer sur cette récolte de données, dans le but de fournir un outil pour l'accélerer et supprimer ses inconvénients.

\subsection{Récolte des données}\label{subsec:recolte-des-donnees}

Les services de la mairie utilisent principalement deux manières de récolter des demandes : les formulaires papiers, et les formulaires en ligne.

\subsubsection{Formulaires papier}

La majorité des demandes sont réalisées via des formulaires papier.
Ceux-ci sont distribués dans les centres civils (mairie, écoles, associations…) pour être trouvés par les administrés.

Le traitement de ces demandes est aujourd'hui majoritairement dématérialisé dans plusieurs logiciels,
les employés doivent donc saisir toutes les demandes papier dans ces logiciels.
Cette saisie pose plusieurs problèmes :
\begin{itemize}
	\item difficulté de relire certaines réponses,
	\item possibilité d'erreur pendant la copie,
	\item temps perdu par la copie,
	\item de nombreuses réponses sont incorrectes (numéro de téléphone à la place d'une addresse mail).
\end{itemize}

\subsubsection{Formulaires en ligne}
\lstset{language=html}

Certains formulaires sont disponibles en ligne sur le site de la ville (\hrefs{ville-arcachon.fr}).
Par exemple, on retrouve :
\begin{itemize}
	\item La demande du telmail\footnote{\hrefs{https://www.ville-arcachon.fr/telmail/}} utilise un formulaire HTML pur créé via WordPress, qui s'intègre bien au reste du site mais dont tous les champs sont de type \lstinline{<input type="text">} (il n'y a donc aucune validation des données remplies).
	\item La demande de carte d'identité ou de passeport\footnote{\hrefs{https://www.ville-arcachon.fr/cni-passeport-rdv-en-ligne/}} utilise un formulaire intégré dans une \lstinline{<iframe>}, qui suit une charte graphique nettement différente (police de caractère, taille du texte, style d'icône, couleur des liens, deuxième scrollbar, etc).
	\item Les formulaires de location de salles\footnote{\hrefs{https://www.ville-arcachon.fr/location-salles/}} sont des PDFs à imprimer, à remplir et à envoyer par la Poste ou par email.
	\item Le Kiosque Arcachon Famille\footnote{\hrefs{https://kaf.ville-arcachon.fr/guard/login}} (garderie…) est sur un site différent, avec une charte graphique très différente.
\end{itemize}

Au delà de l'affichage, le traitement est aussi différent :
\begin{itemize}
	\item Les données provenant des formulaires WordPress sont récupérées via l'interface web d'administration, téléchargées sous le format CSV, importées dans Google Drive pour modifier le format (UTF-8, caractère utilisé pour les colonnes) puis exportée vers Excel pour le traitement par les agents.
	\item Le Kisque Arcachon Famille importe ses données automatiquement dans le logiciel Ciril~\cite{ciril}.
	\item Certains formulaires récoltent leurs données via le logiciel Open Demandes~\cite{opendemandes} (un logiciel libre dont les sources ne sont pas disponibles).
\end{itemize}

\subsection{Excel}\label{subsec:excel}

À l'exception des services utilisant un logiciel métier spécialisé, tous les services importent ou saisissent finalement les données dans Excel.

Cela mène à des fichiers de plusieurs dizaines de milliers de lignes, de rares sauvegardes, et des difficultés de partage (plusieurs agents ne peuvent pas consulter ou modifier le fichier en même temps, transfert par mail lors de communication d'un service à un autre…).

L'utilisation de macros est très répandu, pour automatiser au maximum les tâches de tous les jours (recherche, saisie, etc).
Il est important pour le succès du projet que les workflows réalisés dans Excel puissent être portés vers le nouveau système, mais je n'ai pu en voir qu'un unique exemple.
