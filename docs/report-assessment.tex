\section{Résultats du stage}\label{sec:bilan-du-projet}

Étant donné que la majorité des employés sont en vacances de mi-Juillet à mi-Août, les objectifs suivants avaient été donnés :
\begin{itemize}
	\item Phase 1 (2 Juin au 7 Juin) : Réunions d'analyse des besoins, conception,
	\item Phase 2 (7 Juin au 24 Juin) : Création de groupes, création et saisies de formulaires,
	\item Phase 3 (24 Juin au 17 Juillet) : Gestion des utilisateurs, design graphique, HTTPS, sécurité de la connexion au service,
	\item Phase 4 (19 Juillet au 15 Août) : Afficher les saisies des utilisateurs, recherches dans les saisies, pièces jointes,
	\item Phase 5 (16 Août au 15 Septembre) : Résolution de bugs, optimisation, fonctionnalités non planifiées.
\end{itemize}

Ce planning devait permettre de commencer de présenter le logiciel aux services et d'effectuer leur migration à partir du 15 Août (retours de vacances), tout en laissant suffisamment de temps pour régler les problèmes et ajouter les fonctionnalités proposées lors de cette migration.
Une réunion était organisée environ toutes les deux semaines pour faire un point sur l'avancement.

\uparagraph
Le projet a avancé parfaitement dans les temps jusqu'à la phase 4.
J'avais sous-estimé le temps nécessaire pour les considérations de sécurité (envoi de pièces jointes par les habitants, connexion sécurisée au site), qui ont causé un retard d'environ une semaine pour la recette de la phase 4.

Lors d'une réunion en début de phase 5, il a été estimé que le développement aurait entre une et deux semaines de retard sur la fin du stage.
Il m'a été proposé de passer auto-entrepreneur pour implémenter les dernières fonctionnalités manquantes, et pour régler les problèmes qui arriveraient dans le futur.
Cela m'a permis de finir toutes les fonctionnalités importantes au 16 septembre.

Pour atteindre cet objectif, il a été nécessaire de simplifier certaines fonctionnalités (par exemple, l'affichage de statistiques a été remplacé par un export CSV des données, pour permettre aux employés de générer les statistiques qu'ils souhaitent en utilisant d'autres logiciels).
Néanmoins, le projet était intégré sur le site de la ville moins d'une semaine après la fin du stage.

Depuis cette date, toutes les fonctionnalités nécessaires à l'utilisation par les employés sont implémentées et le travail a majoritairement consisté à la résolution de bugs (dont certains bloquants).
Le compte GitLab a été transféré à la DSI le 19 septembre.
