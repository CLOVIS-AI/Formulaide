%! Author = Ivan Canet
%! Date = 30/07/2021

% Preamble
\documentclass[11pt,french]{memoir}

% Packages
\usepackage{amsmath}
\usepackage{babel}
\usepackage{clovisai}

\newcommand{\hrefs}[1]{\href{#1}{\texttt{#1}}}

% Document
\begin{document}
	\renewcommand{\chaptername}{Partie}

	\frontmatter
	\title{Formulaide --- Stage de 2\ieme{} année à la mairie d'Arcachon}
	\date{1\ier{} Juin -- 15 Septembre 2021}
	\author{Ivan Canet, ENSEIRB-MATMECA}
	\maketitle

	\begin{abstract}
		La Mairie d’Arcachon reçoit chaque jour des centaines de demandes en tout genre provenant des habitants.
		Actuellement, ces demandes arrivent sous format papier, que les employés doivent alors saisir dans un système informatisé (Excel dans de nombreux cas, mais aussi des logiciels métier).
		Ces traitements sont longs, répétitifs et automatisables.

		Il est prévu, dans un futur proche, de refaire entièrement le site de la ville.
		L’objectif de ce stage est de fournir un outil permettant de collecter les demandes via ce nouveau site, pour éliminer le papier et permettre aux habitants de remplir tous les formulaires en ligne.
		Pour que cet outil soit utile, il faut qu’il corresponde à un gain de temps significatif pour la majorité des employés concernés, il est donc important que l’outil n’exige pas une seconde saisie par l’employé pour utiliser son logiciel métier.

		La majorité des services ont un fichier Excel qui correspond à leurs demandes métiers.
		Ces fichiers ne sont pas standardisés, posent des problèmes de sauvegarde, ne peuvent pas être utilisés par plusieurs employés en même temps, et sont difficiles à mettre à jour (par exemple, chaque service a une colonne « habitant », il arrive fréquemment qu’un service soit informé des décès ou déménagements, mais non les autres).
		La mise en place d’une base de données réelle peut régler tous ces problèmes, à condition que cela ne demande pas de modification majeure à la manière de travailler des employés.

		Les services utilisant des logiciels métiers spécialisés sont voués à n’utiliser que très peu l’outil réalisé lors de ce stage.
		Pour ces services, l’objectif est de s’arrêter à l’étape de collecte, en rendant l’import dans leur logiciel le plus simple possible.

		La création et modification de formulaires proposés à la population doit pouvoir être effectuée sans l’intervention d’un développeur.

		\uparagraph
		Le projet consiste donc en la réalisation d’un serveur, fournissant une base de données
		et une API pouvant être utilisée par le nouveau site, ainsi qu’un client léger à l’intention des
		employés.
		Tout cela doit être entièrement transparent pour les habitants (de leur point de vue,
		il s’agit simplement de remplir un formulaire en ligne).
	\end{abstract}

	\tableofcontents

	\mainmatter


	\chapter{Contexte}\label{ch:contexte}


	\section{La mairie d'Arcachon}\label{sec:la-mairie-d'arcachon}

	Arcachon est une commune du Sud-Ouest de la France, sous-préfecture du département de la Gironde, en région Nouvelle-Aquitaine.
	Elle a été créée en 1857 par détachement d'une partie de la commune de La Teste-de-Buch.

	La mairie contient environ 80 agents, répartis entre plusieurs services, ayant à répondre à un grand nombre de problématique (organisation des retraites, état civil, inscriptions scolaires, gestion informatique, etc).
	La majorité de ces services ont en commun de récolter des demandes de la part de la population, le plus souvent sous la forme de formulaires papier.
	Par exemple, on retrouve les demandes de permis de construire, des suggestions pour la vie politique de la ville, l'inscription à la cantine de la garderie, etc.

	Dans ce document, nous allons nous concentrer sur cette récolte de données, dans le but de fournir un outil pour l'accélerer et supprimer ses inconvénients.

	\subsection{Récolte des données}\label{subsec:recolte-des-donnees}

	Les services de la mairie utilisent principalement deux manières de récolter des demandes : les formulaires papiers, et les formulaires en ligne.

	\subsubsection{Formulaires papier}

	La majorité des demandes sont réalisées via des formulaires papier.
	Ceux-ci sont distribués dans les centres civils (mairie, écoles, associations…) pour être trouvés par les administrés.

	La plupart de ces formulaires est créée par le service des Communications, mais certains sont créés par les services qui en sont responsables (par exemple, tous les formulaires en rapport avec la petite enfance).
	Les formulaires ont donc une charte graphique très variée, et des difficultés de distribution existent.

	Le traitement de ces demandes est aujourd'hui majoritairement dématérialisé dans plusieurs logiciels,
	les employés doivent donc saisir toutes les demandes papier dans ces logiciels.
	Cette saisie pose plusieurs problèmes :
	\begin{itemize}
		\item difficulté de relire certaines réponses,
		\item possibilité d'erreur pendant la copie,
		\item temps perdu par la copie,
		\item de nombreuses réponses sont incorrectes (numéro de téléphone à la place d'une addresse mail).
	\end{itemize}

	\subsubsection{Formulaires en ligne}
	\lstset{language=html}

	Certains formulaires sont disponibles en ligne sur le site de la ville (\hrefs{ville-arcachon.fr}).
	Par exemple, on retrouve :
	\begin{itemize}
		\item La demande du telmail\footnote{\hrefs{https://www.ville-arcachon.fr/telmail/}} utilise un formulaire HTML pur créé via WordPress, qui s'intègre bien au reste du site mais dont tous les champs sont de type \lstinline{<input type="text">} (il n'y a donc aucune validation des données remplies).
		\item La demande de carte d'identité ou de passeport\footnote{\hrefs{https://www.ville-arcachon.fr/cni-passeport-rdv-en-ligne/}} utilise un formulaire intégré dans une \lstinline{<iframe>}, qui suit une charte graphique nettement différente (police de caractère, taille du texte, style d'icône, couleur des liens, deuxième scrollbar, etc).
		\item Les formulaires de location de salles\footnote{\hrefs{https://www.ville-arcachon.fr/location-salles/}} sont des PDFs à imprimer, à remplir et à envoyer par la Poste ou par email.
		\item Le Kiosque Arcachon Famille\footnote{\hrefs{https://kaf.ville-arcachon.fr/guard/login}} (garderie…) est sur un site différent, avec une charte graphique très différente.
	\end{itemize}

	Au delà de l'affichage, le traitement est aussi différent :
	\begin{itemize}
		\item Les données provenant des formulaires WordPress sont récupérées via l'interface web d'administration, téléchargées sous le format CSV, importées dans Google Drive pour modifier le format (UTF-8, caractère utilisé pour les colonnes) puis exportée vers Excel pour le traitement par les agents.
		\item Le Kisque Arcachon Famille importe ses données automatiquement dans le logiciel Ciril~\cite{ciril}.
		\item Certains formulaires récoltent leurs données via le logiciel Open Demandes~\cite{opendemandes} (un logiciel libre dont les sources ne sont pas disponibles).
	\end{itemize}

	\subsection{Excel}\label{subsec:excel}

	À l'exception des services utilisant un logiciel métier spécialisé, tous les services importent ou saisissent finalement les données dans Excel.

	Cela mène à des fichiers de plusieurs dizaines de milliers de lignes, de rares sauvegardes, et des difficultés de partage (plusieurs agents ne peuvent pas consulter ou modifier le fichier en même temps, transfert par mail lors de communication d'un service à un autre…).

	L'utilisation de macros est très répandu, pour automatiser au maximum les tâches de tous les jours (recherche, saisie, etc).
	Il est important pour le succès du projet que les workflows réalisés dans Excel puissent être portés vers le nouveau système, mais je n'ai pu en voir qu'un unique exemple.


	\section{Contraintes et objectifs}\label{sec:contraintes-et-objectifs}

	L'objectif de ce stage est de créer un outil permettant au personnel de la mairie de créer des formulaires, de les rendre accessibles sur le site de la ville sans intervention de développeurs, de collecter les réponses, et d'aider au traitement et à l'import dans les logiciels métiers.

	\paragraph{Authentification}
	Il est impératif (vœu politique) que le système ne nécessite \emph{pas} de comptes de la part des administrés.
	Cette volonté provient du nombre élevé de comptes qui sont déjà nécessaires pour accéder aux différents services de la ville.
	De plus, la faible qualité des mots de passe dans la population est un problème de sécurité.

	Une conséquence de cette décision est qu'il n'est pas possible de créer un «~espace utilisateur~» pour suivre les dossiers en cours, ou les modifier lorsqu'une pièce est incorrecte.
	Pour cette raison, en cas de dossier incorrect, l'administré sera obligé de faire une nouvelle saisie (permettre à un employé de modifier la saisie d'un administré a été jugé comme trop dangereux).
	Il devient donc aussi impossible pour les administrés de prouver leur identité auprès du système, les employés devront donc vérifier l'identité des demandeurs pour chaque formulaire.

	\paragraph{Interface publique}
	Lors de ce projet, je ne devais pas être amené à réaliser une interface graphique visible par les administrés.
	Cette contrainte provient des chartes graphiques de la mairie, qui ont un rôle important pour l'identité de la ville.
	D'autre part, je n'ai pas compétences de design.

	Puisque le logiciel produit est destiné à l'utilisation interne par les agents, cette contrainte n'a eu d'impact que sur la partie d'interfaçage avec le site de la ville.

	Pour des raisons similaires, il n'était pas souhaité que le logiciel envoie des accusés de réception aux utilisateurs (par mail ou SMS).
	Lors de la création d'un formulaire, il est donc nécessaire que les employés mettent en place une manière de contacter les administrés (pour demander des informations manquantes, etc).

	\paragraph{Accessibilité}
	L'outil doit être utilisable par les employés de l'accueil de la mairie, pour permettre aux personnes âgées ou autres personnes n'ayant pas d'accès internet de remplir les formulaires.
	Dans ce cas, les employés doivent effectuer la saisie, mais l'on suppose que ce cas est suffisamment rare pour qu'un gain de temps soit effectué sur les autres dossiers.

	L'utilisation d'un logiciel web permet aussi une meilleure accessibilité pour les personnes malvoyantes que les formulaires papier, en particulier grâce aux lecteurs d'écrans.
	HTML5, le standard actuel pour la création de pages web, a été pensé pour être compatible avec ces outils.

	\paragraph{Maintenance}
	Pour faciliter les audits indépendants et la maintenance future, le logiciel entier est publié sous la license Apache 2.0 dans un dépôt public sur GitLab.
	Cette license autorise toute modification pour n'importe quel but, avec très peu de contraintes (attribution…).


	\chapter{Réalisation}\label{ch:realisation}


	\section{Technologies utilisées}\label{sec:technologies-utilisees}

	\subsection{Implémentation}\label{subsec:livrables-et-developpement}

	Le choix des technologies utilisées pour la réalisation du projet s’est basé sur plusieurs critères, que nous allons maintenant détailler.

	\subsubsection{Base de données}

	Comme nous le verrons plus tard, le domaine est composé d’un grand nombre de structures récursives.
	Les bases de données relationnelles ne sont pas en mesure de travailler efficacement avec des données récursives~: les deux options sont d’utiliser une table pour représenter un nœud, mais il faut dans ce cas enchaîner un grand nombre de JOIN, ou alors de sérialiser l’arbre dans une seule colonne, mais il devient alors impossible d’effectuer des recherches sur son contenu.
	Les bases de données relationnelles sont aussi connues pour la complexité des migrations (modifications de la base de données sans suppression des données contenues).
	Puisque le domaine comporte de nombreuses structures compliquées, il était nécessaire d’utiliser des outils étant capable de supporter des modifications du schéma avec le moins de maintenance possible.

	\uparagraph
	MongoDB est une base de données de type NoSQL (non-relationnelle) qui est devenue populaire dans les récentes années grâce à son intégration avec Docker, NodeJS et Kubernetes.
	MongoDB ne requiert pas de schéma, et est donc capable de gérer des ajouts de colonnes ou suppressions de colonnes dynamiquement.
	Au lieu de fournir un langage spécifique au domaine (DSL) comme les bases de données relationnelles (SQL), MongoDB se base sur le format JSON (avec de légères modifications).

	Puisque toutes les colonnes représentent des objets JSON, il est possible de stocker des listes, et de faire de recherche dans les listes à l’intérieur des colonnes.
	Cela permet de bien meilleures performances que ce qui serait possible avec une base de données relationnelles, tout en étant plus simple à utiliser puisque le langage peut travailler avec des colonnes réelles ou des listes JSON dans des colonnes, de manière transparente.

	De manière générale, les bases de données NoSQL essaient de décentraliser les données : permettre à plusieurs répliques de la base de coopérer, et de permettre au système de survivre à l’échec d’une des répliques.
	Ceci est un grand atout pour le client, puisqu’il est très important que le système soit toujours accessible.

	\uparagraph
	MongoDB a aussi la particularité de fournir des implémentations du driver pour de nombreux langages, qui, au lieu d’être un miroir du langage de requêtes, essaient de s’intégrer au mieux dans le langage source.
	Cette intégration permet un développement très rapide, car travailler avec la base de données devient aussi simple que de travailler avec les structures habituelles du langage.

	\subsubsection{Langage et architecture}

	Les besoins du client étant de fournir à la fois une interface web et une API REST, il était nécessaire de choisir des technologies permettant de satisfaire ces besoins.
	Il existe un grand nombre d’alternatives, dont les plus populaires sont JavaScript pour l’interface et PHP pour le serveur.

	En raison de la complexité du domaine, il était nécessaire de ne pas avoir à l’implémenter deux fois~;
	les solutions du type JavaScript + PHP étaient donc à proscrire.
	Peu de langages sont capables de servir à la fois côté serveur et client : les deux principaux sont JavaScript et Kotlin.

	\uparagraph
	JavaScript est un langage originellement développé pour remplacer Java pour les sites internet, dans les années 1990.
	Il a connu une croissance très forte et est aujourd’hui un des langages les plus populaires.
	Récemment, NodeJS a permis à JavaScript d’être utilisé pour la réalisation de serveurs, ce qui permet de réaliser un projet entier en JavaScript.

	Kotlin est un langage créé en 2011 par JetBrains (entreprise développant Android Studio, IntelliJ IDEA, WebStorm, PyCharm, etc) pour remplacer Java dans leurs outils.
	Kotlin garde les concepts majeurs de Java (programmation orientée objet) tout en ajoutant de nouvelles fonctionnalités (programmation fonctionnelle, gestion de la nullabilité au niveau des types, inférence de types, programmation asynchrone, etc).
	En 2017, Google a annoncé le support officiel de Kotlin pour Android, puis en 2019 a annoncé que les APIs Android futures seraient écrites pour Kotlin, et non pour Java.
	Ces annonces ont propulsé la popularité de Kotlin, qui est actuellement un des langages les plus demandés.
	Il s’agit d’un langage originellement basé sur la JVM, mais il peut aussi être transpilé vers JavaScript (pour le web), ou vers des exécutables natifs (via LLVM, pour les appareils connectés et iOS).

	JavaScript est un langage typé dynamiquement (les vérifications de compatibilité des données sont faites pendant l’exécution du programme).
	Kotlin est un langage typé statiquement (les vérifications sont réalisées pendant le développement).
	Cela permet d’éviter un très grand nombre d’erreurs et d’augmenter la qualité du logiciel fini.
	N’ayant jamais utilisé TypeScript, je ne l’ai pas considéré comme alternative.
	À l'inverse, j'ai réalisé de nombreux projets en Kotlin.

	\uparagraph
	Le projet consiste donc en deux applications : un serveur s’exécutant sur la JVM, et une interface web transpilée en JavaScript, s’exécutant dans le navigateur.
	La communication entre ces deux applications est réalisée via HTTP, en transmettant des objets Kotlin sérialisés en JSON via KotlinX.Serialization, en utilisant le framework HTTP Ktor (côté client et côté serveur).

	De cette manière, il est possible de créer un module commun contenant tous les objets de l’API, d’y implémenter toute la logique de vérification des données, puis de le partager entre le client et le serveur.
	Cela permet de s’assurer que le client et le serveur effectuent les mêmes
	opérations.
	Le projet est donc séparé entre les modules :
	\begin{itemize}
		\item \lstinline{api}, qui contient tous les objets du domaine, la logique de validation, etc (JVM \& JS).
		\item \lstinline{client}, qui contient l’implémentation côté client de l’API (dépend de \lstinline{api}, JVM \& JS).
		\item \lstinline{ui}, qui contient l’implémentation de l’interface graphique (dépend de \lstinline{client}, JS).
		\item \lstinline{database}, qui contient la connexion et les requêtes à la base de données (dépend de \lstinline{api}, JVM).
		\item \lstinline{server}, qui contient l’implémentation côté serveur de l’API (dépend de \lstinline{api} et \lstinline{database}, JVM).
	\end{itemize}

	\subsubsection{Interface web}

	React est un framework déclaratif pour créer des interfaces graphiques en JavaScript.
	C’est une des solutions les plus populaires aujourd’hui, ce qui rend la maintenance future bien plus aisée.
	JetBrains (les développeurs de Kotlin) proposent des bindings pour React (informations de types).

	Pour la décoration de la page, le choix de TailwindCSS a été fait.
	TailwindCSS est une collection de classes CSS utilitaires, permettant d’accélérer le développement d’interfaces.
	En utilisant un framework déclaratif comme React, on n’écrit pas directement du HTML, l’utilisation de classes utilitaires permet de placer le style directement dans la structure de l’interface.

	\subsection{Suivi de projet}\label{subsec:suivi-de-projet}

	La gestion de projet est réalisée sous GitLab~\cite{gitlab}.

	GitLab est un logiciel de gestion des sources (basé sur Git), de gestion de projet, de DevOps, etc.
	GitLab est disponible en version Community (gratuite et libre), et plusieurs versions Entreprise (payantes et non-libres).
	Toutes ces versions sont disponibles dans le Cloud (géré par l'entreprise GitLab) ou On-Premises (géré par la Mairie).
	Les versions Entreprise sont gratuites dans un cadre de projet Open Source.
	Puisque le code n'est pas secret, et est Open Source, GitLab Cloud a été choisi pour simplifier l'installation, et faciliter l'accès aux futurs auditeurs ou collaborateurs.

	\subsubsection{Forks}

	GitLab permet l'utilisation de forks, des copies du projet conservant un lien avec la version originale.
	Les forks permettent de laisser des entreprises tierces accéder au code et de modifier leur copie comme elles se souhaitent, tout en fournissant une manière privilégiée pour transmettre des améliorations au projet d'origine.
	Cela permet à la mairie d'avoir le contrôle total sur sa version du logiciel, tout en permettant l'accès aux futurs mainteneurs.

	En suivant ce modèle, j'ai travaillé pendant le stage sur ma propre copie du dépôt de la mairie, tout en publiant toutes les nouvelles versions de manière jointe sur le dépôt de la mairie.
	À la fin du stage, les droits d'administration ont été transférés à Benoît Landa (employé de la DSI).

	\subsubsection{Liste des tâches}

	GitLab permet d'organiser la liste des tâches.
	Cette liste est alors disponible dans IntelliJ (l'IDE que j'utilise), ce qui permet de générer automatiquement une branche qui correspond à la tâche voulue, et de chronométrer le temps passé à la réaliser.
	Ces tâches sont organisées en jalons (milestones) pour avoir une vue globale sur l'avancée du projet et de quantifier les retards.

	Les tâches peuvent posséder plusieurs étiquettes, qui sont utilisées pour communiquer sur la zone du projet impactée, la gravité du problème, et la priorité de la résolution.

	Le Service Desk permet de créer une addresse mail qui transforme chaque mail reçu comme une tâche.
	Cette addresse mail est affichée lors d'un plantage de l'application, ce qui permet aux utilisateurs de signaler les erreurs directement sur GitLab depuis leur boite mail, sans avoir à se créer un compte.

	\subsection{DevOps}\label{subsec:devops}

	GitLab permet d'automatiser de nombreux aspects du développement.
	Par exemple, une nouvelle version du projet est générée automatiquement lorsque l'on crée une étiquette Git dans l'interface web, ce qui permet aux employés de la mairie de créer des versions et de les déployer sans avoir besoin de cloner le dépôt ni d'installer quelque outil que ce soit.

	\subsubsection{Tests et vérification}

	Lors de toute modification du code, les tests suivants sont effectués :
	\begin{itemize}
		\item La compilation des différents modules du projet,
		\item Les tests unitaires et d'intégration des différents modules du projet,
		\item La mesure du code testé (ne mesure que le code Kotlin/JVM à cause de l'indisponibilité d'une technologie similaire pour Kotlin/JS),
		\item L'analyse statique du code grâce à Qodana (uniquement à titre indicatif, Qodana a été jugée trop expérimentale et instable pour être utilisée comme critère d'acceptation du code),
	\end{itemize}

	De plus, un certain nombre de fichiers sont générés :
	\begin{itemize}
		\item Le guide utilisateur et le rapport de stage sont compilés depuis leurs sources en \LaTeX,
		\item La génération du code est publiée en HTML (via l'outil Dokka, un équivalent de Javadoc ou Doxygen),
		\item Les différents livrables du projet (voir la section suivante).
	\end{itemize}
	Les fichiers générés sont ensuite rendus accessibles sur GitLab Pages, un outil permettant de générer des sites statiques directement depuis les pipelines d'automatisation.

	\subsubsection{Livrables}

	Pour simplifier la gestion des dépendances, le projet est livré sous la forme d'un conteneur Docker, ainsi que la configuration nécessaire pour Docker Compose.
	Un conteneur est une manière d'encapsuler un logiciel et ses dépendances dans un environnement séparé de l'OS sur lequel il tourne, mais sans les barrières de sécurité des machines virtuelles.
	Les conteneurs peuvent être utilisés de manière similaire aux machines virtuelles, tout en étant bien plus rapides et plus légers.

	Docker est l'entreprise fournissant l'implémentation la plus populaire de conteneurs.
	Docker Compose est une technologie permettant d'organiser la coopération entre plusieurs conteneurs, qui peut être distribuée sous la forme de quelques fichiers de configuration.
	Grâce à cet outil, il suffit à l'administrateur de récupérer ses fichiers, puis de les exécuter, pour mettre entièrement en place les différents services du projet.

	Dans ce projet, quatre services sont utilisés :
	\begin{itemize}
		\item \lstinline{mongo}, le conteneur officiel fourni par MongoDB (pour faciliter les mises à jour de sécurité),
		\item \lstinline{mongo-express}, une interface web d'administration de MongoDB,
		\item \lstinline{server}, le conteneur fourni par ce projet, contenant le serveur web ainsi que l'interface graphique (automatiquement généré lors de la création d'une nouvelle version, stocké dans le registre du dépôt sur GitLab),
		\item \lstinline{proxy}, le reverse-proxy Caddy qui implémente SSL et renouvelle les certificats automatiquement (le seul conteneur accessible depuis le réseau extérieur).
	\end{itemize}
	En plus de la déclaration de ces conteneurs, la configuration Docker Compose détermine les ports accessibles depuis l'extérieur, le paramétrage des variables d'environnement et des mots de passe d'administration, etc.
	Ces fichiers sont disponibles dans le dossier \lstinline{docker} du dépôt.

	Les fichiers sources sont aussi disponibles sous la forme d'archives (ZIP et TAR.GZ).


	\section{Sécurité}\label{sec:securite}

	Dans cette section, nous allons aborder la sécurité au niveau de la base de données, du serveur,
	et enfin du client.

	\subsection{Base de données et serveur}\label{subsec:base-de-donnees-et-serveur}

	La base de données est la partie la plus critique du logiciel.
	Un attaquant en prenant le contrôle aurait un pouvoir absolu sur le logiciel entier.
	Cependant, la base de données ne nécessite d’être accessible que par le serveur, il est donc possible de l’isoler complètement du monde extérieur.
	Toutes les requêtes à la base de données sont générées par le client officiel MongoDB~\cite{mongo}, via l’intermédiaire de KMongo~\cite{mongo-kotlin} ; il ne devrait donc pas être possible d’effectuer une attaque par injection à travers le serveur.
	La configuration fournie pour Docker n’expose pas la base de données.

	\uparagraph
	Toutes les communications entre les clients et le serveur ont lieu via HTTP (ou HTTPS).
	Le serveur livré n’est pas capable de gérer HTTPS de lui-même, mais la configuration est donnée pour utiliser un reverse-proxy~\cite{caddy}.
	Ce reverse-proxy permet de gérer automatiquement la mise à jour des certificats SSL, etc.
	La publication des ports HTTP (non-sécurisés) du serveur n’est pas nécessaire, il est donc possible de n’exposer publiquement que les ports HTTPS du reverse-proxy.

	La partie de l’API destinée à être utilisée par le site public de la ville est très limitée, et ne contient aucun transfert d’identifiants.
	HTTPS est jugé comme une protection suffisante.
	Au sein de la base de données, les mots de passe sont hachés via BCrypt~\cite{bcrypt}.

	\subsection{Authentification}\label{subsec:authentification}

	L’interface interne pour les employés communique avec le serveur via HTTPS ou HTTP.
	Un avertissement de sécurité est affiché si HTTPS n’est pas disponible.
	Pour des raisons de sécurité, il existe deux identifiants pour chaque utilisateur : un \emph{access token} et un \emph{refresh token}.

	L’\emph{access token} permet de s’identifier auprès du serveur, et d’effectuer n’importe quelle action en tant qu’un employé.
	Il n’est valable que quelques dizaines de minutes, et n’est stocké que dans une variable JavaScript ; il ne devrait donc pas être possible de le voler.
	Le \emph{refresh token} permet de demander au serveur de fournir un nouveau access token, quand le précédent a expiré.
	Il est stocké comme cookie dans le navigateur, protégé par :
	\begin{itemize}
		\item \lstinline{HttpOnly} : inaccessible par le code JavaScript de la page,
		\item \lstinline{Secure} : transmis uniquement via HTTPS,
		\item \lstinline{SameSite Strict} : le navigateur refuse de transmettre le cookie à un autre serveur que celui qui l’a créé.
	\end{itemize}
	Le \emph{refresh token} reste valable pendant quelques jours, mais sa durée de vie est étendue à chaque
	utilisation.
	Si l’utilisateur modifie son mot de passe, tous les \emph{refresh token} existant deviennent invalides (déconnectant de force sur toutes les machines qui avaient accès à son compte).

	\uparagraph
	Les deux attaques courantes sur un site web sont XSS~\cite{xss} et CSRF~\cite{csrf}.

	On appelle une attaque XSS une situation dans laquelle l’attaquant peut exécuter du code JavaScript arbitraire sur une page du site.
	Elles ont souvent lieu grâce à une injection de code.
	L’interface graphique est écrite via ReactJS~\cite{react} et n’utilise pas les fonctions \lstinline{dangerouslySetInnerHtml} ou similaire ;
	elle ne devrait donc pas être sensible aux injections de code.
	Si une attaque XSS a lieu, le refresh token ne sera pas accessible, grâce à \lstinline{HttpOnly} : même si un attaquant arrive à exécuter du code arbitraire, il n’aura pas accès aux identifiants.

	On appelle une attaque CSRF une situation dans laquelle l’attaquant peut exécuter une requête depuis la page (se faisant ainsi passer pour l’utilisateur, du point de vue du serveur).
	Comme les attaques XSS, ces attaques ont souvent comme origine des injections de code.
	Si une attaque CSRF a lieu, l’attaquant n’aura pas accès au refresh token mais sera en possibilité de le transmettre au serveur, ce qui lui permettrait de récupérer un access token qu’il pourrait ensuite utiliser pour prendre le contrôle du compte, jusqu’à son expiration.
	L’attaquant ne pourra pas transmettre le refresh token vers un autre site (grâce à \lstinline{SameSite}), il ne peut donc pas le voler.

	Si l’attaquant a un accès physique au navigateur de la victime, il est trivial d’accéder au cookie et le copier.
	En cas de suspicion, il faut modifier le mot de passe de l’utilisateur concerné.

	\subsection{Mots de passe}\label{subsec:mots-de-passe}

	La modification de mot de passe nécessite :
	\begin{itemize}
		\item de transmettre un \emph{access token} valide,
		\item de fournir le mot de passe actuel (sauf pour l'administrateur).
	\end{itemize}

	Si l’utilisateur ne connait pas le mot de passe de la victime, une attaque CSRF dure donc au maximum le temps de validité d’un access token.
	Une attaque par accès physique dure jusqu’à ce que la victime modifie son mot de passe.
	Il est important de noter qu’une attaque sur un compte administrateur est bien plus grave :
	quelqu’un prenant le contrôle d’un compte administrateur peut modifier tous les mots de passe, et ainsi bloquer tous les employés (dont l’administrateur réel)hors du site.
	Il n’est pas possible de régler cette situation via l’API ni l’interface fournie (sinon l’attaquant y aurait aussi accès)---dans cette situation, la seule solution est d'intervenir sur la base de données directement.

	Il existe une faille théorique lorsqu’un utilisateur modifie son mot de passe plus de $2^{64}$ fois (environ 18 milliards de milliards de fois) pendant moins de temps que la durée de validité d’un \emph{refresh token} (quelques jours, les extensions de durée lors de l’utilisation ne sont pas utilisables dans cette situation).
	Je considère que cette faille ne pose pas de risque crédible qui mériterait de la corriger.

	\uparagraph
	Pour empêcher qu’un attaquant devine le mot de passe d’un utilisateur en essayant des mots de passe communs, deux protections sont en place :
	\begin{itemize}
		\item Le blocage des comptes cibles,
		\item Le blocage du serveur entier.
	\end{itemize}

	Lorsqu’un utilisateur essaie de se connecter avec un mot de passe incorrect, son compte est bloqué pendant quelques secondes, cumulées à chaque échec.
	Un compte utilisateur n’est bloqué au maximum que pendant quelques minutes à partir de la fin de l’attaque.
	Quand un compte est bloqué, il est impossible de s’y connecter, même avec le mot de passe correct (même si l’attaquant devine le mot de passe réel, il ne peut pas se connecter).
	Un utilisateur déjà connecté (qui possède un \emph{refresh token}) n’est pas impacté par le blocage de son compte (les employés peuvent continuer à travailler normalement même pendant une attaque).

	L’interface ne prévient pas l’utilisateur que son compte est actuellement bloqué, pour ne pas donner d’informations à l’attaquant.
	Pour un administrateur, la seule manière de savoir pourquoi l’accès à été refusé est de vérifier les journaux du serveur.

	Une variation de l’attaque par recherche de mot de passe consiste à essayer un même mot de passe envers chaque compte, pour esquiver le blocage d’un compte cible.
	Pour se protéger contre ce type d’attaque, chaque réplique du serveur est capable de se bloquer temporairement (comme si tous les comptes étaient bloqués).
	Le fonctionnement est identique au blocage d’un compte spécifique, mais affecte tous les comptes et dure moins longtemps.
	De la même manière, les employés déjà connectés ne sont pas impactés.
	Les utilisateurs anonymes (les habitants de la ville, qui remplissent des formulaires) ne sont impactés par aucun blocage, puisqu’ils ne possèdent pas de compte.

	\subsection{Bilan de sécurité}\label{subsec:bilan-de-securité}

	En conclusion de ces attaques, les risques les plus importants sont :
	\begin{itemize}
		\item Les mots de passe de mauvaise qualité.
		Quoiqu’il arrive, aucune des mesures de sécurité mise en place n’est efficace contre un attaquant qui connaît déjà le mot de passe.
		Face à une attaque de ce type, la solution est de changer le mot de passe.
		\item Une attaque par accès physique sur un compte employé.
		L’attaque dure au maximum jusqu’à la fin de l’accès physique (ou jusqu’à la modification du mot de passe) plus la durée de vie d’un access token (une dizaine de minutes).
		\item Une attaque par accès physique sur un compte administrateur.
		L’attaque peut potentiellement empêcher tous les employés d’accéder au système.
		La seule solution est une intervention technique sur la base de données, ou remettre en place une sauvegarde datant d’avant l’attaque tout en changeant le secret des tokens.
		\item L’interception du traffic HTTP\@.
		L’attaquant peut avoir accès total à un compte jusqu’à ce que le mot de passe soit modifié.
		Interdir l’accès au serveur via HTTP, et autoriser uniquement HTTPS, rend cette attaque impossible à exploiter.
		\item Une attaque XSS ou CSRF sur un navigateur ne supportant pas toutes les fonctionnalités de sécurité (à supposer qu’une attaque XSS ou CSRF existe).
		Dans ce cas, l’attaquant pourrait avoir accès total à un compte jusqu’à ce que le mot de passe soit modifié.
		À l’écriture de ce document, les navigateurs compatibles~\cite{browser-compat-security} sont Chrome 80 (PC et Android), Edge 86 et Opera 71.
		Firefox 69 est compatible à condition d’activer un paramètre spécifique.
		Internet Exporer est incompatible, quelle que soit la version.
	\end{itemize}

	On notera que toutes ces attaques sont en lien avec les comptes employés ou administrateurs : la seule attaque possible via un accès anonyme (les habitants de la ville) consiste à polluer la base de données avec des données inutiles, il n’est pas possible ni d’accéder ni de modifier les données existantes.
	Une simple protection DoS/DDoS est efficace contre ce type d'attaque (mais n'est pas incluse par la configuration fournie du projet).


	\section{Groupes et formulaires}\label{sec:groupes-et-formulaires}


	\section{Saisie des réponses}\label{sec:saisie-des-reponses}


	\section{Suivi}\label{sec:suivi}


	\chapter{Conclusion}\label{ch:conclusion}


	\section{Bilan du projet}\label{sec:bilan-du-projet}


	\section{Retour sur les choix techniques}\label{sec:retour-sur-les-choix-techniques}


	\section{Bilan personnel}\label{sec:bilan-personnel}

	\appendix
	\renewcommand{\bibname}{Références}
	\begin{thebibliography}{9}

		\bibitem{mongo}
		MongoDB : The application data platform,
		\hrefs{https://www.mongodb.com}

		\bibitem{mongo-java}
		Java MongoDB Driver,
		\hrefs{https://docs.mongodb.com/drivers/java}

		\bibitem{mongo-kotlin}
		KMongo, a Kotlin toolkit for Mongo,
		\hrefs{https://litote.org/kmongo}

		\bibitem{caddy}
		Caddy 2 is a powerful, entreprise-ready, open source web server with automatic HTTPS written in Go,
		\hrefs{https://caddyserver.com}

		\bibitem{bcrypt}
		BCrypt, Wikipedia,
		\hrefs{https://en.wikipedia.org/wiki/Bcrypt}

		\bibitem{xss}
		Cross-Site Scripting (XSS), Wikipedia,
		\hrefs{https://en.wikipedia.org/wiki/Cross-site\_scripting}

		\bibitem{csrf}
		Cross-Site Request Forgery (CSRF), Wikipedia,
		\hrefs{https://en.wikipedia.org/wiki/Cross-site\_request\_forgery}

		\bibitem{react}
		React, a JavaScript library for building user interfaces,
		\hrefs{https://reactjs.org}

		\bibitem{browser-compat-security}
		Compatibilité des navigateurs et fonctionnalités de sécurité des cookies,
		\hrefs{https://developer.mozilla.org/en-US/docs/Web/HTTP/Headers/Set-Cookie\#browser\_compatibility}

		\bibitem{gitlab}
		Code source du projet:
		\hrefs{https://gitlab.com/arcachon-ville/formulaide}

		\bibitem{ciril}
		Ciril,
		\hrefs{https://www.ciril.net/fr/}

		\bibitem{opendemandes}
		Open Demandes,
		\hrefs{https://www.girondenumerique.fr/opendemandes.html},
		\hrefs{https://www.icm-services.fr/solutions/solutions-open-demande}

	\end{thebibliography}

\end{document}
