\section{Suivi}\label{sec:suivi}

Lorsque les administrés remplissent un formulaire, leurs saisies doivent être transmises aux différents services de la mairie.
Pour cela, l'administrateur peut créer plusieurs «~étapes~» lors de la création d'un formulaire.

Une étape représente la vérification de la saisie par un agent de la mairie appartenant à un service particulier.
Ensembles, les étapes forment un automate dont l'état piège correspond aux dossiers refusés.

Chaque étape peut aussi comporter des champs réservés à l'administration, qui sont visibles par les autres employés.
Les employés ont donc la possibilité :
\begin{itemize}
	\item de refuser un dossier (il sera déplacé dans l'étape spéciale «~Dossiers refusés~»),
	\item d'accepter un dossier (il sera transféré au service responsable de l'étape suivante),
	\item de déplacer un dossier vers une étape précédente (lors d'une erreur de saisie),
	\item de conserver le dossier dans l'étape actuelle, en l'annotant (par exemple, si l'étape correspond à prendre un rendez-vous, mais que la personne n'est pas disponible).
\end{itemize}
Le système conserve l'historique de toutes les opérations appliquées à un dossier.

\uparagraph
Du point de vue de la base de données, chaque dossier est une association entre des identifiants de champs et les valeurs saisies par l'administré.
Cela permet de modifier les noms des champs, leur type ou même leur ordre sans risquer de perdre des données.

Pour rendre impossible la perte de données, le système n'est pas capable de modifier les dossiers.
L'inconvénient de cette méthode de fonctionnement est que la suppression de champs dans un formulaire ne permet pas de réellement supprimer les données.
